\RequirePackage[l2tabu, orthodox]{nag} % This gives you warnings if you use obsolete LaTeX commands.
\documentclass[12pt,reqno]{amsart}
\usepackage{fullpage,amssymb,amsbsy,url,enumerate,color,comment,colonequals,graphicx,ifthen,mathrsfs,stmaryrd,bm}
\usepackage[all]{xy}
\usepackage[ruled, linesnumbered]{algorithm2e}

% Color!
\usepackage[dvipsnames,xcdraw,hyperref]{xcolor}
\definecolor{darkgreen}{rgb}{0,0.5,0}
\newcommand{\green}[1]{{\color{darkgreen} #1}}
\newcommand{\blue}[1]{{\color{blue} #1}}
\newcommand{\red}[1]{{\color{red} #1}}
\newcommand{\magenta}[1]{{\color{magenta} #1}}
\newcommand{\violet}[1]{{\color{violet} #1}}

\newcommand{\defi}[1]{\textsf{\color{blue} #1}} % for defined terms

% Characters
\newcommand{\Aff}{\mathbb{A}}
\newcommand{\C}{\mathbb{C}}
\newcommand{\F}{\mathbb{F}}
\newcommand{\G}{\mathbb{G}}
\newcommand{\bbH}{\mathbb{H}}
\newcommand{\N}{\mathbb{N}}
\newcommand{\PP}{\mathbb{P}}
\newcommand{\Q}{\mathbb{Q}}
\newcommand{\R}{\mathbb{R}}
\newcommand{\Sphere}{\mathbb{S}}
\newcommand{\Z}{\mathbb{Z}}
\newcommand{\Qbar}{{\overline{\Q}}}
\newcommand{\Zhat}{{\widehat{\Z}}}
\newcommand{\Zbar}{{\overline{\Z}}}
\newcommand{\kbar}{{\overline{k}}}
\newcommand{\Kbar}{{\overline{K}}}
\newcommand{\Fbar}{{\overline{\F}}}

\newcommand{\boldmu}{\bm{\mu}}
\newcommand{\boldalpha}{\bm{\alpha}}

\newcommand{\ii}{\mathbf{i}}
\newcommand{\jj}{\mathbf{j}}
\newcommand{\kk}{\mathbf{k}}

\newcommand{\pp}{\mathfrak{p}}
\newcommand{\mm}{\mathfrak{m}}

% mathcal characters
\newcommand{\calA}{\mathcal{A}}
\newcommand{\calB}{\mathcal{B}}
\newcommand{\calC}{\mathcal{C}}
\newcommand{\calD}{\mathcal{D}}
\newcommand{\calE}{\mathcal{E}}
\newcommand{\calF}{\mathcal{F}}
\newcommand{\calG}{\mathcal{G}}
\newcommand{\calH}{\mathcal{H}}
\newcommand{\calI}{\mathcal{I}}
\newcommand{\calJ}{\mathcal{J}}
\newcommand{\calK}{\mathcal{K}}
\newcommand{\calL}{\mathcal{L}}
\newcommand{\calM}{\mathcal{M}}
\newcommand{\calN}{\mathcal{N}}
\newcommand{\calO}{\mathcal{O}}
\newcommand{\calP}{\mathcal{P}}
\newcommand{\calQ}{\mathcal{Q}}
\newcommand{\calR}{\mathcal{R}}
\newcommand{\calS}{\mathcal{S}}
\newcommand{\calT}{\mathcal{T}}
\newcommand{\calU}{\mathcal{U}}
\newcommand{\calV}{\mathcal{V}}
\newcommand{\calW}{\mathcal{W}}
\newcommand{\calX}{\mathcal{X}}
\newcommand{\calY}{\mathcal{Y}}
\newcommand{\calZ}{\mathcal{Z}}

\newcommand{\CC}{\mathscr{C}}
\newcommand{\FF}{\mathscr{F}}
\newcommand{\GG}{\mathscr{G}}
\newcommand{\scriptH}{\mathscr{H}}
\newcommand{\II}{\mathscr{I}}
\newcommand{\JJ}{\mathscr{J}}
\newcommand{\KK}{\mathscr{K}}
\newcommand{\LL}{\mathscr{L}}
\newcommand{\OO}{\mathscr{O}}
\newcommand{\XX}{\mathscr{X}}
\newcommand{\ZZ}{\mathscr{Z}}

% various math expressions that shouldn't be italicized 
\DeclareMathOperator{\cis}{cis}
\DeclareMathOperator{\var}{var}
\DeclareMathOperator{\Var}{Var}
\DeclareMathOperator{\Cov}{Cov}
\DeclareMathOperator*{\lcm}{lcm}
\DeclareMathOperator{\ord}{ord}
\DeclareMathOperator{\Ker}{Ker}
\DeclareMathOperator{\Bin}{Bin}
\DeclareMathOperator{\res}{res}
\DeclareMathOperator{\rad}{rad}
\DeclareMathOperator{\Spec}{Spec}

\newcommand{\Cech}{\v{C}ech}
\newcommand{\del}{\partial}
\newcommand{\directsum}{\oplus} % binary direct sum
\newcommand{\Directsum}{\bigoplus} % direct sum of a collection
\newcommand{\injects}{\hookrightarrow}
\newcommand{\intersect}{\cap} % binary intersection
\newcommand{\Intersection}{\bigcap} % intersection of a collection
\newcommand{\isom}{\simeq}
\newcommand{\HH}{{\operatorname{H}}}
\newcommand{\HHcech}{{\check{\HH}}}
\newcommand{\HHat}{{\hat{\HH}}}
\newcommand{\notdiv}{\nmid}
\newcommand{\surjects}{\twoheadrightarrow}
\newcommand{\tensor}{\otimes} % binary tensor product
\newcommand{\Tensor}{\bigotimes} % tensor product of a collection
\newcommand{\To}{\longrightarrow}
\newcommand{\union}{\cup} % binary union
\newcommand{\Union}{\bigcup} % union of a collection

%Mau's
\newcommand{\ra}[1][]{\xrightarrow{#1}}
\newcommand{\f}[2]{\frac{#1}{#2}}
\DeclareMathOperator{\id}{id}
\newcommand{\be}{\mathbf{e}}
\newcommand{\bd}{\mathbf{d}}
\newcommand{\eps}{\varepsilon}
\DeclareMathOperator{\Stab}{Stab}
\DeclareMathOperator{\Span}{Span}
\DeclareMathOperator{\im}{im}
\DeclareMathOperator{\sgn}{sgn}

%%% \numberwithin{equation}{section}
%%% \newtheorem{theorem}[equation]{Theorem} 
%%% etc.

% We use one numbering for all theorems, lemmas, etc.
% Use \newtheorem{theorem}{Theorem}[section] to number by section, so that the first theorem in Section 3 would be Theorem 3.1.
\newtheorem{theorem}{Theorem}
\newtheorem{lemma}[theorem]{Lemma}
\newtheorem{corollary}[theorem]{Corollary}
\newtheorem{proposition}[theorem]{Proposition}

\theoremstyle{definition}
\newtheorem{definition}[theorem]{Definition}
\newtheorem{question}[theorem]{Question}
\newtheorem{conjecture}[theorem]{Conjecture}

\theoremstyle{remark}
\newtheorem{example}[theorem]{Example}
\newtheorem{examples}[theorem]{Examples}
\newtheorem{remark}[theorem]{Remark}
\newtheorem{remarks}[theorem]{Remarks}

\newenvironment{problem}
	{
		\bigskip
		\noindent
		{\large\textbf{Problem.}}
		\newline
	}
	{
		\bigskip
	}

\newenvironment{solution}[1]
	{
		\bigskip
		\noindent
		{\large\textbf{Solution to #1.}}
		\newline
	}
	{
		\ensuremath{\blacksquare} \bigskip
	}

\usepackage{microtype}  % This adjusts spacing between words so as to improve the probability of having line breaks in good places.

\usepackage[
%	draft,
%	colorlinks,
%	pagebackref,
%	pdfauthor={Authors go here}, % not necessary to enter this
%	pdftitle={Paper title}, % not necessary to enter this
]{hyperref} % This allows you to put hyperlinks in your document.

% \begin{center}
%    \includegraphics[width=0.5\textwidth]{Filename}
% \end{center}
\usepackage[utf8]{inputenc}

\title{Tetrahedra UROP Project 2}
\author{Mauricio Barba da Costa}
\date{September 2020}

\begin{document}

\maketitle
\section{Introduction}
A tetrahedron is a polyhedron with 4 faces. Suppose $T$ is a tetrahedron with angles
$\alpha_{12},\alpha_{13},\alpha_{14},\alpha_{23},\alpha_{24},\alpha_{34}$. 
My team and I were curious about tetrahedra whose angles, when viewed as elements of 
$\R/2\pi\Q$, span a 5-dimensional $\Q$ vector space. In particular, we wanted to know
if one such tetrahedra could have Dehn invariant zero. Showing that no such tetrahedron
with this property have Dehn invariant zero would have serious ramifications. For instance, Debrunner showed
that if a polyhedron $P$ tiles 3D space then it must have Dehn invariant zero.
\section{The Problem}
Given a tetrahedron $T$, enumerate the vertices 1,2,3,4. Denote $e_{ij}$
as the edge between vertices $i$ and $j$ and $\theta_{ij}$ as the dihedral
angle of $e_{ij}$. One convention that I follow throughout this paper
is listing edges and angles using the ordering $12,13,14,23,24,34$.
Suppose we're given dihedral angles $\theta_{12},...,\theta_{34}\in \R/\Q\pi$ 
that span a 5-dimensional $\Q$-vector space. 
For a tetrahedron with such angle to have Dehn invariant 0, the edge 
lengths $e_{12},...,e_{34}$ must be proportional to a tuple of positive
integers such that $\sum_{i<j} e_{ij}\theta_{ij}=0 \mod{\Q\pi}$. Let $z_{ij}=e^{i\theta_{ij}}$.
Let $w_{ij}=e^{i(2\theta_{ij})}$. Then $2\cos\theta_{ij}=z_{ij}+z_{ij}^{-1}$
and $2\cos2\theta_{ij}=w_{ij}+w_{ij}^{-1}$. By Theorem 1 of Wirth-Dreiding,
$\cos\theta_{ij}=\frac{D_{ij}}{\sqrt{D_{ijk}D_{ijl}}}\in \sqrt{\Q}$
and consequently $\cos2\theta_{ij}=2\cos^2\theta_{ij}-1\in \Q$. Then $w_{ij}$
satisfies the polynomial relation $w_{ij}^2-c_{ij} w_{ij}+1=0$ where
$c_{ij}=2\cos2\theta_{ij}$.
\section{Numerical Approach}
The code I developed as part of this project is distributed across 3 branches of my forked
repository of Abdelatiff Chentouf Anas' original work. It includes my p-adic approach
to resolving the problem, my numerical approach, and some functions that were generally essential for
making computations with tetrahedra. 

Suppose $\sum_{i<j} e_{ij}\theta_{ij}=0\mod{\Q\pi}$. Then $\sum e_{ij}\theta_{ij}=q\pi$
for some $q\in \Q$ so $\sum_{i<j} e_{ij}\theta_{ij} i=iq\pi$. Then $\prod z_{ij}^{e_{ij}}$ is a root
of unity in a field obtained by adjoining square roots to $\Q$. Now,
$Gal(\Q(\zeta_n/\Q)=(\Z/n\Z)^\times$ so $\zeta_n\in \Q(\sqrt{\Q})$
if and only if $(\Z/n\Z)^\times$ 
is an elementary 2-group. This forces $(\Z/n\Z)^\times=\prod_{p|n}(\Z/p^{e_p}\Z)^\times$
where $e_2\leq 3,e_3\leq 1$ and $e_p=0$ for all other primes $p$. Thus,
$n|24$. Thus, $(\prod_{j=1}^6 w_{ij}^{e_{ij}})^{24}=1$. For a tetrahedra $T$, we can 
calculate $W=\prod_{j=1}^6 w_{ij}^{e_{ij}})^{24}$ with a computer and see if it falls
within some $\epsilon$ of 1. To identify a counterexample, I iterated over randomly generated
sextuples, checked if they determined a tetrahedron according to Lemma 4 of Wirth-Dreiding.
I had some setbacks when I undertook this approach. For a while, I was applying the conditions
for Lemma 4 of Wirth-Dreiding into my computer incorrectly. When I did realize that I was
doing it wrong, I had to end my AWS EC2 instance because my free plan was running out.
It might be worth further exploring this approach.
\section{p-adic Approach}
If we think about p-adics now, if $\sum_{i<j}e_{ij}\theta_{ij}=0\mod{\Q\pi}$ then $\sum e_{ij}v_p(w_{ij})=0$
for every prime $p$. Now, $c_{ij}\in \Q$ so 
computing $v_p(c_{ij})$ is easy. Calculating, $v_p(w_{ij})$
is more difficult to calculate since it's in a field extension of $\Q$.
We can use the following tool:
$$
v_p(w_{ij})=\begin{cases}
  0&v(c_{ij})\geq 0\\
  \pm v(c_{ij})&v(c_{ij})<0
\end{cases}
$$
If $v_p(w_ij)<0$, whether $v_p(w_{ij})=+v(c_{ij})$ or $v_p(w_{ij})=-v(c_{ij})$ we can't know.
\section{My algorithm}
My algorithm exploits the p-adic approach to the problem to find tetrahedra
whose angles span a 5-dimensional $\Q$-vector space and have Dehn invariant zero. 
It does this by randomly generating sextuples, checking if they determine a 
tetrahedra (using Lemma 4 of Wirth-Dreiding). If it is a tetrahedron, then it calcualtes the denominator
of $c_{ij}$ when it is expressed in simplest terms. Recall that
$$
c_{ij}=2\cos2\theta_j=4\cos^2\theta_j-2=\frac{4D_{ij}^2-2D_{ijk}D_{ijl}}{D_{ijk}D_{ijl}}
$$
Now, the denominator can be expressed as 
$$
L_{ij}=\frac{D_{ijk}D_{ijl}}{\gcd(D_{ijk}D_{ijl},4D_{ij}^2-2D_{ijk}D_{ijl})}=
\frac{D_{ijk}D_{ijl}}{\gcd(D_{ijk}D_{ijl},4D_{ij}^2)}
$$
Then, it gets the prime factors of all these. Then, it
finds the valuations of all of these with respect to all the primes. We
can arrange these $L$s into a list $[L_{12},L_{13},L_{14},L_{23},L_{24},L_{34}]$.
Taking the valuation with respect to $p$ of these yields
$$
v_p(L_{ij})=\begin{cases}
  0&v_p(c_{ij})\geq 0\\
  \pm v_p(c_{ij})&v_p(c_{ij})<0
\end{cases}
$$
If $\sum_{i<j}\pm e_{ij}v_p(L_{ij})=0$ for some combination of pluses and minuses,
this is a necessity for the condition to hold. It is not a sufficiency since we don't know
which combination of pluses and minuses is right. Here is how I checked if 
$\sum_{i<j}\pm e_{ij}v_p(L_{ij})=0$ computationally:

For every prime $p$
\begin{enumerate}
  \item Identify all $L_{ij}$ such that $v_p(L_{ij})>0$. Suppose there are $m$ such 
  $L_{ij}$s. Insert these valuatioins into an $1\times m$ matrix $VL$. 
  \item Create an $2^m\times m$ matrix $M$ where the elements in the $i$th row
  are the digits of the $i$th integer (starting from 0) when expressed in base 2.
  \item Perform an elementwise multiplication $M\star VL$ then the matrix
  multiplication $(M\star VL)\times E$ where
  $$
  E=\begin{pmatrix}
    e_{12}\\e_{13}\\e_{14}\\e_{23}\\e_{24}\\e_{34}
  \end{pmatrix}
  $$
  \item Use np.any($(M\star VL)\times E==0$) to see if any combination
  of pluses and minuses yielded the correct result.
\end{enumerate}
Not unless np.any($(M\star VL)\times E==0$) for all primes is the tetrahedron a candidate
for having Dehn invariant zero. This seems to never happen.
\section{Some Observations and Analysis}
I have a \href{https://github.com/mauriciobarba/AC-18.Tetra/tree/main}{GitHub Repo}
where the code for this project is stored. In here is also a file that's the result
of running the p-adic analysis of my code. Here is an example output:\\

\begin{verbatim}
[39, 82, 56, 47, 56, 90]
[2314830, 4265533440, 36952226532, 384102810, 3407333959107, 796296384]
193 [0, 0, 0, 1, 1, 1]
2 [1, 13, 2, 1, 0, 6]
3 [1, 3, 2, 7, 6, 2]
5 [1, 1, 0, 1, 0, 0]
7 [1, 1, 0, 1, 0, 0]
73 [1, 0, 1, 0, 1, 0]
13 [0, 0, 2, 1, 3, 1]
19 [0, 1, 1, 0, 0, 1]
151 [1, 0, 1, 0, 1, 0]
29 [0, 1, 1, 0, 0, 1]
primes passed: {3}
\end{verbatim}
This is how you interpret the results:
\begin{enumerate}
  \item On the first line is randomly generated sextuple of integers that denote the edges of a tetrahedron
  in the order that we've been using $(e_{12},e_{13},e_{14},e_{23},e_{24},e_{34})$.
  \item In the second line is the list $[L_{12},L_{13},L_{14},L_{23},L_{24},e_{34}]$.
  \item In the lines below that, the number in the first column is a prime that divides
  some $L_{ij}$. 
  \item The primes that ``passed" are those such that there exists a combination 
  of pluses and minuses such that $\sum_{i<j}\pm e_{ij}v_p(L_{ij})=0$. 
\end{enumerate}

The Erdos-Kac Theorem states that the probability distribution of 
$$
\frac{\omega(n)-\log\log n}{\sqrt{\log\log n}}
$$
where $\omega(n)$ is the number of distinct
prime factors of $n$
is the standard normal distribution. With this, we can calculate the average
number of distinct prime factors of a number with 9 digits to be 3
with standard deviation approximately 1.6 (need to cite
and check if 1.6 is right). $L_{ij}$
is usually around 9 digits and has more than 3 distinct prime factors.
For instance, from looking at the table above, 
$$
384102810=193\times 2\times 3^7\times 5\times 7\times 13
$$
However, the primes that comprise 384102810 tend to be smaller and have a 
greater power. 
By contrast 1,000,000,003=$23\times 307\times 141623$. This suggests that the 
prime factorizations of the $L_{ij}$s is somewhat anomalous. Further exploring why
this is might bear some fruit. 

The most common valuation list is [1,1,0,1,0,0]
or some tetrahedral rotation of it. Also, you never get [2,0,0,0,0,0,2] or any
tetrahedral rotation of this.
Which valuation lists are possible? I looked at valuation lists mod 2. I found that only a 
few valuation lists are possible mod 2. They are
$$
  [1,1,0,0,1,1]
$$
$$
  [1,1,0,1,0,0]
$$
$$
  [0,0,0,0,0,0]
$$
Maybe we can prove the claim
by showing that for every tetrahedron, there exists a prime $p$ that 
yields a valuation list [1,1,0,1,0,0] (or some tetrahedral rotation of it). Note that it's not possible that 
$\pm e_{12}\pm e_{13}\pm e_{23}=0$ otherwise this would lead to a degenerate tetrahedron because
one of the faces violates the triangle inequality.

In general, my findings support the conjecture that there are no tetrahedra whose angles
span a 5-dimensional $\Q$ vector space. The output of my code showed that seldom does
a prime pass. 
\end{document}